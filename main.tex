\documentclass[12 pt, russian]{article}
\usepackage[T1]{fontenc}
\usepackage{ulem}
\usepackage[utf8]{luainputenc}
\usepackage{geometry}
\usepackage[pdftex]{graphicx}
\geometry{verbose,tmargin=3cm,bmargin=3cm,lmargin=3cm,rmargin=3cm}
\usepackage{amstext}
\usepackage{amsthm}
\usepackage{amssymb}
\usepackage{amsmath}
\usepackage[T1,T2A]{fontenc}
\usepackage[utf8]{inputenc}
\usepackage[english,main=russian]{babel}
\usepackage{setspace}
\usepackage{esint}
\usepackage{comment}
\usepackage{babel}
\usepackage{float}
\usepackage{amsfonts}
\usepackage{fullpage} 
\usepackage{parskip} 
\usepackage{tikz} 
\usepackage{indentfirst}
\title{Курсовая работа Шигина Глеба Сергеевича}
\raggedbottom
\begin{document}
\thispagestyle{empty}
\newtheorem{Thm}{Теорема}[section]
\newtheorem{Lemm}{Лемма}[section]
\newtheorem{Rem}{Замечание}[section]
\newtheorem{Co}{Следствие}[section]
\theoremstyle{definition}
\newtheorem{Exam}{Пример}[section]
\newtheorem{Dfn}{Определение}[section]
\sloppy
\begin{titlepage}
\begin{center}
Московский государственный университет имени~М.~В.~Ломоносова\\
факультет космических исследований\\
кафедра математической статистики и случайных процессов\\
\centering
\includegraphics[width=0.3\textwidth]{mechmath.jpg}

\vspace*{100pt} Курсовая работа\\
студента 402 группы \\
Шигина Глеба Сергеевича \\
\vspace{10pt} {\Large{\textbf{}}\\}
Трансформация данных в GLM моделях
\vspace*{40pt}

\begin{flushright}
Научный руководитель:\\ с.н.с., к.ф.-м.н.\\  Шкляев Александр Викторович\\
\end{flushright}

\vspace*{\fill} Москва, 2022
\end{center}
\end{titlepage}
\section{Введение}
Ветвящиеся процессы были введены Гальтоном и Ватсоном в работе~\cite{Galton}. Классическая теория для этих процессов предполагает, что производящая функция процесса $\phi(s)$ отлична от $s$. Мы дополним исследование недостающим случаем $\phi(s)=s$. Для такой модели мы получим асимптотику вероятностей умеренных, больших и сверхбольших уклонения ветвящегося процесса.

Итак, ветвящимся процессом называют цепь Маркова $Z_n$, такую что $Z_0=1$, а
$$
Z_{n+1} = \sum_{i=1}^{Z_n} \xi_{n,i},
$$
где $\xi_{n,i}$ --- независимые одинаково распределенные (н.о.р.) случайные величины (сл.в), имеющие производящую функцию $\phi(s)=s$. Основной результат работы заключается в нахождении асимптотики вероятностей
$$
{\bf P}(Z_n \ge x) 
$$
при различных $x$ и $n\to\infty$.

Работа устроена следующим образом: в разделе~\ref{Before} содержатся предварительные сведения, которые потребуются в ходе доказательства основной теоремы~\ref{MainTh}, содержащейся в разделе~\ref{Main}. Доказательство основной теоремы приведено в разделе~\ref{Proof}. Заключительные замечания сделаны в разделе~\ref{Final}.
\section{Предварительные сведения}
\label{Before}
Рассмотрим ветвящийся процесс Гальтона-Ватсона, определенный выше. Тогда нам потребуются следующие факты:
\begin{Lemm} (Лемма 1.1(iv), \cite{Branch})
Производящая функция $Z_n$ задается соотношением
$$
\phi_{Z_n}(s) = \phi_n(s),
$$
где $\phi_n(s)$ --- $n$-кратная суперпозиция производящей функции $\phi$.
\end{Lemm}
\begin{Co}
Предположим, что $\phi(s)=s$. Тогда $\phi_{Z_n}(s) = s$.
\end{Co}
\section{Основной результат}
\label{Main}
Итак, предположим, что ветвящийся процесс $Z_n$ имеет производящую функцию числа потомков одной частицы $\phi(s)=s$. Тогда справедлива следующая теорема, включающая себя анализ умеренных, больших и свербольших уклонений процесса $Z_n$:
\begin{Thm}
\label{MainTh}
Пусть $k>1$. Тогда
$$
\lim_{n\to\infty} {\bf P}(Z_n\ge k) = 0,
$$
причем сходимость равномерна по рассматриваемым $k$.
\end{Thm}
\section{Доказательства}
\label{Proof}
\begin{proof}[Доказательство теоремы~\ref{MainTh}]
Доказательство очевидно.
\end{proof}
\section{Заключение}
\label{Final}
В работе получено решение значимой проблемы, на протяжении долгих лет остававшейся не исследованной. Отметим, что проблема малых уклонений и по сей день остается открытой и требует дальнейших исследований.
\bibliographystyle{elsarticle-num-names}
\bibliography{BiblioIvanov}
\end{document}
