\documentclass[12 pt, russian]{article}
\usepackage[T1]{fontenc}
\usepackage{ulem}
\usepackage[utf8]{luainputenc}
\usepackage{geometry}
\usepackage[pdftex]{graphicx}
\geometry{verbose,tmargin=3cm,bmargin=3cm,lmargin=3cm,rmargin=3cm}
\usepackage{amstext}
\usepackage{amsthm}
\usepackage{amssymb}
\usepackage{amsmath}
\usepackage[T1,T2A]{fontenc}
\usepackage[utf8]{inputenc}
\usepackage[english,main=russian]{babel}
\usepackage{setspace}
\usepackage{esint}
\usepackage{comment}
\usepackage{babel}
\usepackage{float}
\usepackage{amsfonts}
\usepackage{fullpage} 
\usepackage{parskip} 
\usepackage{tikz} 
\usepackage{indentfirst}
\title{Курсовая работа Шигина Глеба Сергеевича}
\raggedbottom
\begin{document}
\thispagestyle{empty}
\newtheorem{Thm}{Теорема}[section]
\newtheorem{Lemm}{Лемма}[section]
\newtheorem{Rem}{Замечание}[section]
\newtheorem{Co}{Следствие}[section]
\theoremstyle{definition}
\newtheorem{Exam}{Пример}[section]
\newtheorem{Dfn}{Определение}[section]
\sloppy
\begin{titlepage}
\begin{center}
Московский государственный университет имени~М.~В.~Ломоносова\\
Факультет космических исследований\\
Кафедра математической статистики и случайных процессов\\
%\centering
%\includegraphics[width=0.3\textwidth]{mechmath.jpg}

\noindent\rule{16cm}{0.4pt}
\vfill

\vspace*{100pt} Курсовая работа\\
студента 402 группы \\
Шигина Глеба Сергеевича \\
\vspace{10pt} {\Large{\textbf{}}\\}
Трансформация данных в GLM моделях
\vspace*{40pt}

\begin{flushright}
Научный руководитель:\\ с.н.с., к.ф.-м.н.\\  Шкляев Александр Викторович\\
\end{flushright}

\vspace*{\fill} Москва, 2022
\end{center}
\end{titlepage}
\section{Введение}
\label{sec:Intro}
TBA
\section{Трансформация данных}
\label{sec:Theory}
Существуют классы задач, для которых мы знаем, что матожидание ${\bf E}(Y|X)$ является линейной функцией от $X$. 
Это может быть теория, подкрепленная экспериментальными данными \cite{PhysRevD.17.2875}, либо статистическая информация о данных. Например,
пусть $y_i$ и $x_i$ -- выборки их нормальных распределенний со средними $\mu_X,\mu_Y$ соответственно, дисперсиями $\sigma_X,\sigma_Y$ соответственно и
корреляцией $\rho_{XY}$. Тогда можно показать (см. \cite[стр. 550]{Berger2001-pm}), что 
\begin{align}
\label{eq:normal-normal}
y_i\mid x_i \sim N\left( \mu_Y-\rho_{XY}\frac{\sigma_Y}{\sigma_X}\mu_X + \rho_{XY}\frac{\sigma_Y}{\sigma_X}x_i,\sigma_Y(1-\rho^2_{XY}) \right).
\end{align}
Это можно переписать как $y_i\mid x_i \sim N \left (\beta_0 + \beta_1 x_i,\sigma^2 \right)$, где 
\begin{align}
\beta_0 = \mu_Y-\rho_{XY}\frac{\sigma_Y}{\sigma_X}\mu_X, \beta_1 = \rho_{XY}\frac{\sigma_Y}{\sigma_X},\sigma^2 = {\bf D}(Y|X) = \sigma^2_Y (1-\rho_{XY}),
\end{align}
то есть мы получили линейную регрессию $Y$ по $X$:
\begin{align}
    {\bf E}(Y|X=x_i) = \beta_0 + \beta_1 x_i.
\end{align}
Однако, у нас не всегда есть возможность узнать истинную зависимость целевой переменной от предикторов, 
и любое преобразование данных, которое мы используем, не более чем приблежение, которое,
как мы надеемся, является подходящим для рассматриваемой задачи. Это поднимает два важных вопроса:
еак выбрать преобразование и подходит ли полученная модель к имеющимся данных?

Для удобства на данном этапе ограничимся одним \textit{предиктором} $X$ и \textit{зависимой переменной} $Y$.
\subsection{Трансформация зависимой переменной методом обратной функции}
Предположим, что истинная регрессионная модель $Y$ по $X$ имеет вид:
\begin{align}
    \label{eq:g-regression}
    Y = g(\beta_0 + \beta_1 X + \varepsilon),
\end{align}
где $g$ -- некоторая функция, вообще говоря, нам неизвестная. Модель (\ref{eq:g-regression}) может быть приведена
к линейному виду, преобразовав $Y$ с помощью обратной функции $g^{-1}$:
\begin{align}
    g^{-1}(Y) = \beta_0 + \beta_1 X + \varepsilon.
\end{align}
Например, если $Y = \log (\beta_0 + \beta_1 X + \varepsilon),$ то $g(x) = \log (x) ,$ значит $g^{-1}(x) = \exp (x)$, и $\exp(Y) = \beta_0 + \beta_1 X + \varepsilon.$

Существует несколько способов получения оценки $g^{-1}$, например, с помощью графика обратного отклика
(\textit{inverse response plot}) \cite{Cook-Weisberg}, или с помощью подбора функции из степенного семейства (\textit{power family}), 
включающим в себя семейство преобразований Бокса-Кокса \cite{Box-Cox-1964}. В данной работе мы сосредоточимся на последнем методе.

\section{Заключение}
\label{sec:End}
TBA

\bibliographystyle{apalike}
\bibliography{Biblio}
\end{document}
